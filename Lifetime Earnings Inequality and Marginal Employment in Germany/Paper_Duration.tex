\documentclass[12pt, a4paper]{article}
\usepackage[german, english]{babel}
\usepackage[T1]{fontenc}
\usepackage[utf8]{inputenc}
\usepackage{bbm}
\usepackage{lscape}
\usepackage{rotating}
\usepackage{tablefootnote} 
\usepackage{footnote}
\usepackage[bottom]{footmisc}
\usepackage{graphicx}
\usepackage[section]{placeins}
\usepackage{apacite}
\usepackage{pdflscape}
\usepackage{booktabs,xcolor,siunitx}
\newcommand\doubleRule{\toprule\toprule}
\newcommand\doublerule{\toprule\specialrule{\heavyrulewidth}{\doublerulesep}{0.95em}}
\definecolor{lightgray}{gray}{0.95}
\usepackage{lmodern}
\usepackage{babel,blindtext}
\usepackage{hyperref}
%\PassOptionsToPackage{bookmarks={false}}{hyperref}
%\PassOptionsToPackage{draft}{hyperref}
%\hypersetup{draft}

\hypersetup{pdfstartview={XYZ null null 1.00}}
\makeatletter \setlength{\@fptop}{0pt} \makeatother
\hypersetup{
	colorlinks,
	linkcolor={red!70!black},
	citecolor={blue!70!black},
	urlcolor={blue!80!black}
}
\usepackage[update]{epstopdf}
\usepackage[left=1.25in,right=1.25in,top=1in,bottom=1.5in]{geometry}\usepackage{amsmath,amssymb,amsthm,amsfonts,dsfont,color}
\usepackage[mathscr]{euscript}
\usepackage{multirow}
\usepackage{natbib}
\usepackage[onehalfspacing]{setspace}
\usepackage[footnotesize,justification=centering,singlelinecheck=off]{caption}
\usepackage{subcaption}
\usepackage{ragged2e}
\usepackage{cleveref}
\usepackage{floatflt}
\usepackage{subfloat}
\usepackage{wrapfig}
\usepackage{changepage}
\usepackage{eurosym}
\usepackage{lscape}
\usepackage{threeparttable}
\usepackage[utf8]{inputenc}
\usepackage{authblk}
\usepackage{parskip}
\parskip 0pt
\usepackage{titling}
\setlength{\droptitle}{-2em}
\usepackage{abstract}
\setlength{\absleftindent}{0mm}
\setlength{\absrightindent}{0mm}
\usepackage[capposition=top]{floatrow}

\newtheorem{thm}{Theorem}
\newtheorem{lem}{Lemma}

\newcommand{\Keywords}[1]{\par\noindent
{\small{\em Keywords\/}: #1}}
\DeclareMathOperator*{\argmax}{arg\,max}
\renewcommand{\thefigure}{\Roman{figure}}
%\renewcommand{\thesubfigure}{(\alph{subfigure})}
\hypersetup{bookmarksdepth=-2}

\begin{document}
\date{\today}
\begin{titlepage}


\title{\textbf{Lifetime Earnings Inequality and Flexible Working Time Arrangements}}

\author{Manuel Sanchez\\
Preliminary\thanks{
Sanchez: University of Bristol, The Priory Road Complex, Bristol, UK BS8 1TU, +44 (0117)3310702, ms15072@bristol.ac.uk.}.}

\maketitle

\vspace{-0.5 cm}
\begin{abstract}
\vspace{+0.5 cm}
\noindent
To be written.

\vspace{+0.5 cm}
\textbf{Keywords}: Panel data, life-cycle, earnings dynamics, unobserved heterogeneity.\\
%\textbf{JEL}: C23; D91; E21; J31.\\
%\textbf{Link}: \url{https://www.dropbox.com/s/p4d23jwi06trlma/ModellingOnline.pdf?dl=0}.
\end{abstract}

\thispagestyle{empty}
\end{titlepage}

\newpage
\setcounter{page}{1}

\setlength{\parindent}{10pt}
\section{Research Proposal R-SIAB 7514} 

This year, individuals entering the labor market in Germany are going to experience larger lifetime earnings inequality than their predecessors. \cite{BoCo15} document a striking secular rise of intergenerational inequality in lifetime earnings: West German men born in the early 1960s are likely to experience about 85\% more lifetime inequality than their fathers. This increase in lifetime earnings inequality has happened both at the upper tail and at the bottom tail of the earnings distribution, although the rise has been larger at the bottom. 

%A recent paper by

\sloppy{The greater amount of intragenerational inequality in lifetime earnings that prospect workers will endure matters for the social welfare of generations. Given the weight that earnings have in lifetime income, larger intragenerational inequality of lifetime earnings is indicative of larger inequality of permanent incomes. Since the standard of living depends more on lifetime income than on any given year's income, studies about inequality of the standard of living should focus on the distribution of lifetime income instead of relying on the distribution of cross sectional income.}

In line with the dramatic rise of intragenerational inequality in lifetime earnings in Germany, there has been a rapid escalation of atypical work arrangements for newer cohorts entering the labor market: (i) the fraction of part-time employees has increased in a linear fashion from 11 percent in 1985 to 22 percent in 2010, (ii) the share of workers with working time accounts rose between 1998 to 2005 from 33 to 48 percent, and (iii) the number of marginal part-time work (`mini-jobs') has increased in recent years, reaching the second most important form of employment with around half of the establishments using some type of marginal employment\footnote{OECD Data, \cite{Gro2006}, \cite{Bechmann10}, \cite{CarriTuRo15}, \cite{Galasi16}.}.

%, which is prevalent in the German labor market,

This growth in alternative work arrangements has mixed implications for workers. On one hand, flexible working time contributes to the decentralization of employment relations on multiple levels, decreasing the bargaining power of workers and individualizing working conditions. This is often seen as a more precarious and lower paid work than regular open-ended employment\footnote{\cite{AddisonCotti15}, \cite{DuSpOe14}.}. On the other hand, flexible working time serves as an alternative to forced redundancies and can be used by employees as a mean to insure against unemployment and labor income risk. Workers under marginal employment can use their flexible working time to embark on training programs, having more time and being better prepared to search for stable jobs. In this way, flexible work arrangements would act as as a stepping stone to permanent work\footnote{\cite{Berg2008}, \cite{Seifert2006}, \cite{Caliendo2016}, \cite{Booth02}.}.

%Flexible working time practices have the advantage of allowing for the substitution of future time off for immediate increases in pay. However, workers bear the loss of income for the expansion of hours without receiving the otherwise overtime pay premiums.

%The decline of unionized workers is seen as a key driver of the growth of wage inequality at the bottom of the earnings distribution.

%and labor supply shocks -immigration waves (East of Germany)-

At the moment, not much is known about the impact that new, flexible forms of employment have on lifetime earnings inequality.  \cite{BoCo15} show that the increase has been stronger at the bottom of the distribution and they find that 20\%-40\% of the rise of lifetime inequality can be attributed to an increase in the duration of unemployment. The rest is due to an increase of intragenerational wage inequality. However, their data set does not contain information about the attachment of workers to the labor market, so they cannot distinguish between the role played by the inequality in hourly wages and the one played by the inequality in hours worked. The way in which flexible employment statuses may enter an explanation of the rise of intragenerational lifetime earnings inequality is a priori unclear and merits an in-depth investigation.

%Hence, it remains an open question whether flexible working arrangements generate larger lifetime earnings inequality

%While lifetime earnings inequality has increased both at the bottom and at the top of the distribution, little is known about its sources.

We want to study how more flexible forms of labor contracts affect the distribution of lifetime earnings. For that, we aim at targeting the German labor market, as it is a paradigm in introducing flexible forms of employment such as marginal employment. As opposed to cross sectional studies, we want to opt for a cohort-based analysis of the mobility experienced over the life-cycle as it can help understand the consequences of introducing more flexible forms of work. Thus, our goal is to develop a methodology targeting actual lifetime earnings using an employment transition and wage mobility model that captures positional earnings mobility, employment risk, and that allows for an assessment of how different forms of employment affect the degree of inequality in lifetime earnings.

%and the ways in which labor markets have changed during the past decades

Studies on lifetime earnings are scarce since only a few countries provide access to sufficiently long panel data sets. The German R-SIAB 7514 data set is one of those exceptions. The sample size is considerably large and earnings are precisely measured. The data set provides actual information on lifetime earnings for prime aged workers, along several birth cohorts, facilitating the construction of measures of lifetime earnings and reducing the necessity of simulating lifetime values. It also provides unique information regarding the intensity of attachment of workers to the labor market, as in full-time work, part-time work, marginal employment or unemployment spells, something that is key to quantify the contribution of employment statuses on the propagation of lifetime earnings inequality. Finally, panels are sufficiently long, which is a necessary condition to avoid the bias that results from omitting unobserved heterogeneity in employment continuity\footnote{\cite{AddiSur08}, \cite{BowRo10}.}. Thus, incorporating unobserved heterogeneity favors the distinction in the contribution to the rise in lifetime earnings inequality of the increase in the variability of wages both in the short and in the long run or the growing variability of the individual-specific components of earnings.

We obtain the following findings...

%Bargaining decisions become more decentralized with a higher decline of union and employer association membership. 

%alternative work arrangements and

%\cite{Merkl13} show that about 1.5 million or 3.8\% of workers in Germany were on short-time work schemes at the peak of the Great Recession in May 2009. However, \cite{Boy10} show that this form of employment is enacted as a fiscal stabilizer and has virtually been used in a constant fraction of workers for every recession in Germany since 1960.

%Manuel is currently working on a research proposal to extend his access to the German administrative dataset. The idea is to study the effects of marginal employment (the so-called minijobs) through the lens of a statistical model of the joint dynamics of earnings and employment. THis will be the basis of his second chapter (and perhaps job market paper).

%\footnote{}.

%The larger amount of intragenerational inequality of lifetime earnings that prospect workers will confront is important not only to the extent that individuals tend to compare their economic status against their age-specific peers\footnote{\cite{PeAs2011}.}. Given the prominence of earnings as a determinant of the lifetime resources available to agents, larger intragenerational inequality of lifetime earnings is suggestive of larger inequality of permanent incomes. In turn, larger inequality of permanent incomes speaks of larger consumption inequality and has implications to the social welfare of future generations.

%

%We exploit data on earnings biographies from social security records to shed light on the following issues: How does marginal employment affect intragenerational lifetime earnings inequality? To what extent is the rise in intragenerational inequality driven by hours worked? 

\section{Model}\label{sec:Model}

\subsection{Main framework}\label{sec:main_framework}

We follow $N$ workers, indexed $i=1,...,N$, and consider a point process \mbox{$\textbf{x}=(x_{t})_{t\in\mathbb{R}^+}$} for any finite subset $(t_1,...,t_p)$ of $\mathbb{R}^+$ satisfying $t_1<t_2<...<t_p$, $p\geq1$. In our data, we observe the length of individual employment states, daily wages and worker characteristics. Define the right-continuous counting process $M(t)$. The process $M(t)$ makes a jump of unit size at each time there is a transition event for a worker. A typical observation for any individual $i$ will be represented as a vector $\textbf{x}_i=(\textbf{w}_i,\textbf{e}_i,c_i^v,c_i^u)$, where:
\vspace{0.2 cm}
\begin{itemize}
\setlength{\itemsep}{0.2 cm}
\item $\textbf{w}_i=(w_{t}^i)_{t\in\mathbb{R}^+}$ is the observed sequence of individual $i$'s log daily real wage flows.
\item $\textbf{e}_i=(e_{t}^i)_{t\in\mathbb{R}^+}$  is individual $i$'s observed sequence of employment states valued in the finite state space $E=\{FT,PT,ME,CE,NE\}$. $FT$ stands for full-time, $PT$ stands for part-time, $ME$ stands for marginal employment, $CE$ stands for concurrent employment, and $NE$ stands for non-employment. 
\item $c_i^v$ is the observed sequence of individual $i$'s characteristics, where we consider labor market experience.
\item $c_i^{u}$ is the observed unvarying set of individual $i$'s characteristics. It includes education, gender and labor market cohort\footnote{Labor market cohort is understood as the year in which the individual first entered the labor market. Therefore, $c_i^v$ is deterministic conditional on $c_i^u$.}.
\end{itemize}
\vspace{0.2 cm}

%wages or
%unemployment and employment, transitions between full-time, part-time and marginal employment, 

In addition to the observed individual heterogeneity captured by $c_i^v$ and $c_i^{u}$, we allow for unobserved heterogeneity which may influence wages and/or the selection of employment types. Hence, we append a set $k_i$ of (time-invariant) unobserved characteristics. Our goal is to estimate simultaneously transitions between different types of employment and earnings trajectories within and between employment states. We define the likelihood of $(\textbf{x}_i,k_i)$ as follows:

%We assume that unobserved heterogeneity, $k_i$, only affects incomes through its impact on mobility

\begin{align}\label{eq:Likelihood}
\mathcal{L}_{i}(\textbf{x}_i,k_i)=& \Bigg( \frac{1}{\sqrt{2\pi \sigma_{w}^{2}}} \text{exp} \Big\{ -\frac{(\ln w_{i} - \widehat{\ln w_{i}})^{2}}{2\sigma_{w}^{2}} \Big\} \Bigg) \cdot \ell(\textbf{e}_i|\textbf{c}_i^v,c_i^u,k_i^e)\cdot \ell(k_i|c_i^u)\cdot \ell (c_i^u).\\ \nonumber
\end{align}

%Before the last term, $\ell(c_i^v|c_i^u)$ represents the distribution of $c_i^v$, which may depend on $c_i^u$\footnote{For example, education may affect the distribution of individuals working in different sectors.}.

The above likelihood for the typical individual is decomposed into four terms. Starting from the right, $\ell (c_i^u)$ is the sample distribution of observed individual fixed characteristics. This distribution is observed and independent of any parameter. Next, $\ell(k_i|c_i^u)$ is the distribution of unobserved heterogeneity given observed characteristics $(c_i^u,c_i^v)$\footnote{Remember $c_i^v$ is deterministic conditional on $c_i^u$}. Before the last term,  $\ell(\textbf{e}_i|\textbf{c}_{i}^{v},c_{i}^{u},k_{i}^{e})$ is the contribution to the likelihood function from job spells. Finally, the first component is the likelihood contribution from a sequence of wage observations over job spells. Individual $i$'s contribution to the complete likelihood comes from three components, pertaining to wage history, labor market status history and unobserved heterogeneity. In the next subsections, we provide further details of each component of Equation \ref{eq:Likelihood}.

\subsection{Unobserved Heterogeneity}\label{sec:unobs_het}

We consider two types of unobserved heterogeneity, $k_i=(k_i^e,k_i^w)$. The first type, $k_i^e$, relates to heterogeneity in terms of the propensity to be in each employment state (called mobility classes hereafter). $k_i^e$ conditions the parameters relating to employment state history. The second type, $k_i^w$, relates to heterogeneity in terms of wage (called wage classes hereafter) through its impact on wage levels. $k_i^w$ conditions the parameters relating to wage distribution\footnote{This type of heterogeneity increases the persistence of income ranks, which is found to be underestimated otherwise. See \cite{Sh76}.} . Both types of heterogeneity are time-invariant individual random effects, which are allowed to be correlated in an arbitrary fashion.

We refer to wage and mobility classes because we use a finite mixture approach to
model unobserved heterogeneity where an individual can belong to one of $K^{e}$ classes of mobility and $K^w$ wage classes\footnote{The total number of classes is hence $K=K^{e}\times K^{w}$}. The probability of belonging to any latent class depends on observed individual heterogeneity $(c_i^v,c_i^u)$, as follows:

\begin{align}\label{eq:prob_unobs_het}
\text{Pr}\{k_i^e,k_i^y|c_i^u\}=\text{Pr}\{k_i^y|k_i^e,c_i^u\}\cdot\text{Pr}\{k_i^e|c_i^u\}.
\end{align}

The previous two components are modeled as multinomial logits:

\begin{align}\label{eq:prob_unobs_het_multin}
\text{Pr}\{k_i^e=k^e|c_i^u\}=\frac{exp(c_{i,k^e}^{u})}{\sum\limits_{k=1}^{K^e}exp(c_{i,k}^{u})} \,\,\,\,\, \text{and} \,\,\,\,\, \text{Pr}\{k_i^w|k_i^e,c_i^u\}=\frac{exp\Big[\begin{bmatrix}
           c_{i,k^w}^{u} \\
           k_{i,e}
         \end{bmatrix}(k_{k^w}^w)\Big]}{\sum\limits_{k=1}^{K^w}exp\Big[\begin{bmatrix}
           c_{i,k^w}^{u} \\
           k_{i,e}
         \end{bmatrix}(k_{k^w})\Big]}
\end{align}

with $(K^e,K^w)$ outcomes and $(k^e_1,k^w_1)$ taken as the reference category.

\subsection{Labor Market States}\label{sec:lab_market_states}

Transition probabilities between the five distinct labour market states are assumed to depend only on the individual's state at the previous spell and on observed and unobserved heterogeneity - i.e. labour market states are assumed to follow a (conditional) first order Markov chain. The transition probabilities, \mbox{$p_{j,l}(s,s+t)=\text{Pr}(e_{s+t}=l|e_{s}=j)$}, are defined for any $s$ and $t \in \mathbb{R}^+$, and for any $j$ and $l \in E$. In addition, our Markov process is time homogeneous: $p_{j,l}(s,s+t)=p_{j,l}(0,t)=p_{j,l}(t)$. That is, the transition probability does not depend on the origin of the time set, but on the difference between the two dates $s$ and $(s+t)$\footnote{See \cite{FlFoMou07} for more details.}. 

Corresponding to each of these point processes $\textbf{e}_i$ we assume there is a finite stochastic intensity $\lambda_{j,l}^{i}(t)$, for all transitions between states $j,l \in E$. We allow for transitions in the same employment state for the four categories of labor market states; that is, we allow for job-to-job transitions in a given state. An example is a worker who jumps from one firm to another while she is in Full Time employment. The previous intensity describes the instantaneous probability of individual $i$ experiencing a labor market transition from state $j$ to $l$ at time $t$ conditional on the information available before time $t$. We adopt a proportional hazard specification:

\begin{align}
\lambda_{j,l}^{i}(t|c_i^v,c_i^u,k_i^e) = H_{j,l}(t)\cdot\text{exp}\{\beta' c_i^{v} + \gamma' c_i^u\}\cdot k_i^{e}.
\end{align} 

The function $H_{j,k}(t)$ is the baseline hazard, that we assume to be constant; $H_{j,k}(t)=1$\footnote{Thus, our only source of duration dependence will be unobserved heterogeneity in mobility classes.}. The systematic part, $\text{exp}\{\beta' c_i^{v} + \gamma' c_i^u\}$, contains varying and unvarying covariates. Finally, $k_i^{e}$ represents a latent fixed factor.

Define the multi-state feature of the model as a set $\mathbb{Q}$ of transition types $\mathbb{Q} = \{1,2,...,Q\}$ for all $j,l \in E$. For example, $q = 1$ denotes a $FT$ to $PT$ transition, $q = 2$ denotes a $FT$ to $ME$ transition, ..., $q = 6$ denotes a $PT$ to $FT$ transition, $q = 7$ denotes a $PT$ to $PT$ transition, ..., up to $q = \mathbb{Q} = 24$ that denotes a transition from $NE$ to $CE$.  The contribution to the likelihood function from a job spell conditional on the initial states, pre-sample event histories and heterogeneity can be written as

\begin{align}\label{eq:Likelihood_states}
\ell(\textbf{e}_i|\textbf{c}_i^v,c_i^u,k_i^e)  = & \prod_{m=1}^{M(T)} \prod_{i=1}^{N} \prod_{q=1}^{Q} \text{exp}\Big\{\mathbbm{1}_{i,t_m}\{q\} \text{ln}\{\lambda_{q,i}(t_m|c_i^v,c_i^u,k_i^e)\} \\ \nonumber
& - \int_{t_m-1}^{t_m}\lambda_{q,i}(t|c_i^v,c_i^u,k_i^e)dt \Big\}.\\ \nonumber
\end{align}

The likelihood function (\ref{eq:Likelihood_states}) has an intuitive interpretation: the likelihood contains the probability of survival of individual $i$ in its current state over each spell of the pooled point process if there is not a transition for this individual. When the transition $m$ takes place at the end of the spell of the pooled process for individual $i$, that is if $\mathbbm{1}_{i,t_m}\{s\}$, the survival probability is multiplied by the hazard rate to yield the probability density of the transition event.

\subsection{Wage Process}\label{sec:income_process}

We specify the following equation for the log real daily wage at all employment spells $j \in E - \{NE\}$:

\begin{align}
\ln w_{i,t_m}^{j} & = \alpha' c_i^{v} + \delta' c_i^u + \eta_{w}w_{i,t_{m-1}} + k_i^w + \epsilon_{i}^{j} .\\ \nonumber
\end{align}

$k_i^{w}$ is the unobserved heterogeneity which is assumed to be constant across repeated spells. $\epsilon_i^j$ is an error term assumed to be normally distributed with mean zero and unknown variance $\sigma_{w}^{2}$. In addition, we allow the log wage to vary with respect to the wage of the previous employment spell $w_{i,t_{n-1}}$.

\subsection{Likelihood Maximization}\label{sec:maximization}

Since we do not know the unobserved characteristics for an individual $i$, the unconditional log-likelihood contribution corresponds to the weighted sum of contributions corresponding to the $(K^{w},K^{e})$ points of support. The sample log-likelihood is given by:

\begin{align}\label{eq:Compl_likelihood}
\ln \mathcal{L} = \sum_{i=1}^{N} \ln \sum_{k_i^e=1}^{K^e} \sum_{k_i^w=1}^{K^w} \mathcal{L}_{i}[\textbf{x}_i, (k_i^{e},k_i^{w})], 	\\ \nonumber
\end{align}

where individual random effects $k_i = (k_i^e,k_i^w)$ are integrated out of the complete likelihood \ref{eq:Compl_likelihood}. We obtain parameter estimates by maximizing the above log-likelihood function.




\newpage
\addcontentsline{toc}{section}{References}
\bibliographystyle{apalike}
%\bibliographystyle{ecta}
\bibliography{References} 

\newpage
\setcounter{section}{0}
\renewcommand{\thesection}{\Alph{section}} 
\renewcommand{\thesubsection}{\thesection.\arabic{subsection}}
\section{Appendix\label{Sec: Preliminary}}
\setcounter{table}{0}
\renewcommand{\thetable}{A\arabic{table}}
\setcounter{figure}{0}
\renewcommand{\thefigure}{A\arabic{figure}}



\end{document}