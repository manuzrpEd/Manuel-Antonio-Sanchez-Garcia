\documentclass[hyperref={bookmarks=false}]{beamer}
%
% Choose how your presentation looks.
%
% For more themes, color themes and font themes, see:
% http://deic.uab.es/~iblanes/beamer_gallery/index_by_theme.html
%
\mode<presentation>
{
  \usetheme{default}      % or try Darmstadt, Madrid, Warsaw, ...
  \usecolortheme{default} % or try albatross, beaver, crane, ...
  \usefonttheme{default}  % or try serif, structurebold, ...
  \setbeamertemplate{navigation symbols}{}
  \setbeamertemplate{caption}[numbered]
\setbeamertemplate{footline}[frame number]
} 

\newcommand{\itemset}{\setlength{\itemsep}{1.5ex}}

\newcommand*\MyPitem{%
  \item[\color{green}\scalebox{1.1}{+}]}
\newcommand*\MyCitem{%
  \item[\color{red}\scalebox{1.3}{-}]}

\newcommand\doubleRule{\toprule\toprule}
\newcommand\doublerule{\toprule\specialrule{\heavyrulewidth}{\doublerulesep}{0.95em}}

\mode<presentation>
\usepackage[english]{babel}
\usepackage[utf8x]{inputenc}
\usepackage{threeparttable} %allows for table notes
\usepackage{multirow}
\usepackage{tablefootnote} 
\usepackage{hyperref}
\usepackage[center]{caption}
\usepackage{amsmath}
\usepackage{apacite}
\usepackage{natbib}
\usepackage{color}
\usepackage[space]{grffile}
\usepackage{graphicx}
\usepackage{dsfont}
\usepackage{fancyvrb}
\usepackage{eurosym}
\captionsetup[table]{aboveskip=0pt}
\captionsetup[table]{belowskip=-10pt}
%\usepackage{footnote}

\usepackage{enumitem}
\setenumerate[1]{label=\arabic*.}
\newcounter{ResumeEnumerate}

\setitemize{label=\usebeamerfont*{itemize item}%
  \usebeamercolor[fg]{itemize item}
  \usebeamertemplate{itemize item}}

\hypersetup{
%	colorlinks,
%	linkcolor={green!70!black},
	citecolor={blue!70!black},
	urlcolor={blue!80!black}
}

\let\oldcite=\cite                                                   
\renewcommand{\cite}[1]{\textcolor[rgb]{.0,.2,.7}{\oldcite{#1}}}

\title{Long-term Earnings Inequality and Part-Time Work in Germany}
%\author{Manuel Sanchez and Felix Wellschmied}
%\institute{University of Bristol \\ PhD Economics Seminar}
\author[Short Name (U ABC)]{%
  \texorpdfstring{%
    \begin{columns}
      \column{.5\linewidth}
      \centering
     %Manuel Sanchez \\ \scriptsize{University of Bristol}
Manuel Sanchez
      %\column{.5\linewidth}
      %\centering
      %Felix Wellschmied \\ \scriptsize{Universidad Carlos III de Madrid}
    \end{columns}
 }
 %{Author 1, Author 2}
{Author 1}
}
%\institute[short]{XXII WORKSHOP ON DYNAMIC MACROECONOMICS}
\institute{University of Bristol}
%\date{July 4-6, 2017}
\date{\today}

\begin{document}

%\begin{frame}
%  \titlepage
%\end{frame}


\section{Literature Review}
\subsection{}

\begin{frame}
\center{\Huge{Literature Review}}
\end{frame}

\begin{frame}{Literature Review - Lifetime Earnings Inequality}
\cite{BoCo15}
\begin{itemize}
\item Lifetime Earnings inequality in West Germany has increased for male cohorts 1935-1969, VSKT records
\item Both at the bottom and at the top of the distribution, but the rise has been
stronger at the bottom
\item Proximate causes:
\begin{enumerate}
\item Longer unemployment spells of workers at the bottom of the
distribution of younger cohorts contribute to explaining 20\%–40\%
of the overall increase in lifetime earnings inequality.
\item 60\%–80\%
of the overall increase is due to an increase of intragenerational wage inequality
\end{enumerate}
\end{itemize}
\textbf{What I do:} ``The question they do not address is how part-time and marginal employment becoming more popular have affected long time earnings. I show that
higher persistence of flexible forms of employment, plus lower wages in full-time,
both contribute to the increase in bottom earnings inequality.''
\end{frame}

\begin{frame}{Literature Review - Full Time Wage Inequality}
\cite{DuLuSc09}
\begin{itemize}
\item Wage inequality in West Germany has increased over
1975-2004, RSIAB7504 records, Cross-section
\item 1980: $\uparrow$ Top Inequality
\item 1990: $\uparrow$ Top + Bottom Inequality
\item Top Inequality $\leftarrow$ Technological Change (Polarization)
\item Bottom Inequality $\leftarrow$ Supply Shocks + Labor Market Institutions ($\downarrow$ Unionization)
\item GSOEP lower inequality than RSIAB
\item Method: Decomposition Analysis
\end{itemize}
\textbf{What I do:}``I show that the period 1999-2014 displays
an increase in bottom inequality. Most of this increase in inequality happens at
the bottom of the distribution for males in full-time.''
\end{frame}


\begin{frame}{Literature Review - Marginal Employment, Worker Flows}
\cite{CarriTuRo15}
\begin{itemize}
\item $\uparrow$ Total Employment = $\uparrow$ Part Time Employment + $\uparrow$ Concurrent Employment 
\item Mini Jobs + Full Time = $\uparrow$ in concurrent employment
\item $\uparrow$ Marginal Employment = Full Timers taking up a marginal job as a secondary form of employment
\item The majority of workers in exclusive Marginal Employment have a non-participation to mini-job cycle
\item Method: Descriptives + Multinomial logit
\end{itemize}
\textbf{What I do:}``I extend their work by putting a higher emphasis to wage dynamics, combining the earnings and mobility process into an empirical model and looking into the permanent lifetime earnings consequences of these different employment dynamics.''
\end{frame}


\begin{frame}<0>
\addtocounter{framenumber}{-1} 
\scriptsize
\bibliographystyle{apalike}
\bibliography{References}
\end{frame}

\end{document}
