\documentclass[hyperref={bookmarks=false}]{beamer}
%
% Choose how your presentation looks.
%
% For more themes, color themes and font themes, see:
% http://deic.uab.es/~iblanes/beamer_gallery/index_by_theme.html
%
\mode<presentation>
{
  \usetheme{default}      % or try Darmstadt, Madrid, Warsaw, ...
  \usecolortheme{default} % or try albatross, beaver, crane, ...
  \usefonttheme{default}  % or try serif, structurebold, ...
  \setbeamertemplate{navigation symbols}{}
  \setbeamertemplate{caption}[numbered]
\setbeamertemplate{footline}[frame number]
} 

\newcommand{\itemset}{\setlength{\itemsep}{1.5ex}}

\newcommand*\MyPitem{%
  \item[\color{green}\scalebox{1.1}{+}]}
\newcommand*\MyCitem{%
  \item[\color{red}\scalebox{1.3}{-}]}

\newcommand\doubleRule{\toprule\toprule}
\newcommand\doublerule{\toprule\specialrule{\heavyrulewidth}{\doublerulesep}{0.95em}}

\mode<presentation>
\usepackage[english]{babel}
\usepackage[utf8x]{inputenc}
\usepackage{threeparttable} %allows for table notes
\usepackage{multirow}
\usepackage{tablefootnote} 
\usepackage{hyperref}
\usepackage[center]{caption}
\usepackage{amsmath}
\usepackage{apacite}
\usepackage{natbib}
\usepackage{color}
\usepackage[space]{grffile}
\usepackage{graphicx}
\usepackage{dsfont}
\usepackage{fancyvrb}
\usepackage{eurosym}
\captionsetup[table]{aboveskip=0pt}
\captionsetup[table]{belowskip=-10pt}
%\usepackage{footnote}

\usepackage{enumitem}
\setenumerate[1]{label=\arabic*.}
\newcounter{ResumeEnumerate}

\setitemize{label=\usebeamerfont*{itemize item}%
  \usebeamercolor[fg]{itemize item}
  \usebeamertemplate{itemize item}}

\newcommand{\indep}{\rotatebox[origin=c]{90}{$\models$}}

\hypersetup{
%	colorlinks,
%	linkcolor={green!70!black},
	citecolor={blue!70!black},
	urlcolor={blue!80!black}
}

\let\oldcite=\cite                                                   
\renewcommand{\cite}[1]{\textcolor[rgb]{.0,.2,.7}{\oldcite{#1}}}

\title{Hours, Occupations, and Gender Differences in Labor Market Outcomes}
%\author{Manuel Sanchez and Felix Wellschmied}
%\institute{University of Bristol \\ PhD Economics Seminar}
\author[Short Name (U ABC)]{%
  \texorpdfstring{%
    \begin{columns}
      \column{.2\linewidth}
      \centering
     A. Erosa \\
%Manuel Sanchez
      \column{.2\linewidth}
      \centering
      L. Fuister \\
      \column{.2\linewidth}
      \centering
      G. Kambourov \\
      \column{.2\linewidth}
      \centering
      R. Rogerson \\
    \end{columns}
 }
 %{Author 1, Author 2}
{Author 1}
}
%\institute[short]{XXII WORKSHOP ON DYNAMIC MACROECONOMICS}
%\institute{Manuel Sanchez - University of Bristol}
%\date{July 4-6, 2017}
\date{July 2017}

\begin{document}

\begin{frame}
  \titlepage
\end{frame}

% Uncomment these lines for an automatically generated outline.
%\begin{frame}{Outline}
%  \tableofcontents
%\end{frame}

\section{Simple Benchmark Model}
\subsection{}

\begin{frame}
\center{\Huge{Simple Benchmark Model}}
\end{frame}

\begin{frame}{Empirical Facts}
\begin{enumerate}
\item Substantial Negative relationship between the Mean Hours worked
in an occupation and the Dispersion of Hours worked in that occupation.
\end{enumerate}
\begin{figure}[!t]
\centering
\begin{minipage}[b]{0.9\textwidth}{}
\centering
\includegraphics[width=1\textwidth]{Mean_Disp_Hours.png}
\end{minipage}
\end{figure}
Robust across Gender, Age, Time, Education, Intensive (Weekly Hours) and Extensive ($\#$ Weeks) Margin
\setcounter{ResumeEnumerate}{\value{enumi}}
\end{frame}

\begin{frame}{Empirical Facts}
\begin{enumerate}[start=\numexpr\value{ResumeEnumerate}+1]
\item Hourly Wages decline on average as we move from the High-Mean-Low-Dispersion occupations towards the Low-Mean-High-Dispersion occupations
\end{enumerate}
\begin{figure}[!t]
\centering
\begin{minipage}[b]{0.9\textwidth}{}
\centering
\includegraphics[width=1\textwidth]{Mean_Hourly_Wage.png}
\end{minipage}
\end{figure}
\end{frame}

\begin{frame}{Simple Benchmark Model}
\begin{itemize}
\setlength{\itemsep}{0.35 cm}
\item Simple static (cross-sectional) benchmark model of occupational choice and hours worked
\vfill
\begin{itemize}
\setlength{\itemsep}{0.35 cm}
\item Standard 2 occupation Roy model
\item Occupational Comparative Advantage (productivity) to sort across occupations
\end{itemize}
\item Extensions:
\vfill
\begin{itemize}
\setlength{\itemsep}{0.35 cm}
\item Time allocation decision; individuals value leisure
\item Heterogeneity in preferences for leisure
\item Non-convex mapping of \underline{hours worked} Vs \underline{supply of efficiency units of labor}\\
\vfill
Generates positive effect of hours worked on wages\\
Desired Hours of work will bias occupational choices
\end{itemize}
\end{itemize}
\end{frame}

\begin{frame}{Model}
\begin{itemize}
\setlength{\itemsep}{0.35 cm}
\item Individuals: $i\in[0,1]$
\begin{equation}
u_i = u_{i}(c_i,l_i)=u(c_i,T-h_i)=\ln c_i + \phi_i \frac{(T-h_i)^{1-\gamma}}{1-\gamma}
\end{equation}
\item Occupations: $j=1,2$
\item Productivities/Abilities: $(a_{i,1},a_{i,2})$
\item Heterogeneity: $(a_{i1},a_{i2},\phi_{i})\sim$ multivariate log-normal distribution
\item Technology/Output: $Y_{j}=A_{j}E_{j}$ for $j=1,2$
\item $E_j$: Aggregate input of efficiency units of labor in occupation $j$
\item $A_j$: Price of an efficiency units of labor in occupation $j$
\end{itemize}
\end{frame}

\begin{frame}{Model}
\begin{itemize}
\setlength{\itemsep}{0.35 cm}
\item Individual efficiency units: $e_{ij}=a_{ij}h_{ij}^{1+\theta_{j}}$
\item If $\theta_j=0$ we have the standard case in which the supply of efficiency units by an individual is linear in hours worked
\item Allow $\theta_j$ to differ by occupation: assume $\theta_1=\theta>\theta_2=0$
\end{itemize}
\end{frame}

\begin{frame}{Competitive Equilibrium}
\begin{itemize}
\setlength{\itemsep}{0.35 cm}
\item Normalize $P_Y=1$ and $A_j=1$ for $j=1,2$
\item Individual Opt. Problem:
\begin{equation}
\underset{c,h_j,\mathbb{I}}{\text{max}}\,\,\,\, \Big\{\ln c + \phi \frac{(T-[\mathbb{I}h_1+(1-\mathbb{I})h_2)^{1-\gamma}}{1-\gamma} \Big\}
\end{equation}
s.t.
\begin{equation}
c=\mathbb{I}a_{1}h_{1}^{1+\theta_{1}}+(1-\mathbb{I})a_{2}h_{2}^{1+\theta_2},\,\,\,\, \mathbb{I}\in\{0,1\}, \,\,\,\, h_j\in[0,T]
\end{equation}
\end{itemize}
\end{frame}

\begin{frame}{Qualitative Properties of Equilibrium, Hours}
\begin{itemize}
\setlength{\itemsep}{0.35 cm}
\item If the individual chooses to work in occupation $j$, $h_{j}^{\star}$ satisfies
\begin{equation}
\frac{1+\theta_j}{\phi}=h_j(T-h_j)^{-\gamma}%\equiv g(h_j)
\end{equation} 
\begin{figure}[!t]
\centering
\begin{minipage}[b]{0.6\textwidth}{}
\centering
\includegraphics[width=1\textwidth]{Hours_Choice.png}
\end{minipage}
\end{figure}
\end{itemize}
\end{frame}

\begin{frame}{Qualitative Properties of Equilibrium, Hours}
\begin{enumerate}
\setlength{\itemsep}{0.35 cm}
\item $h_j^{\star}\,\,\,\, \indep\,\,\,\, a_j$: an increase in productivity has an income and a substitution effect on labor supply that offset because of the specification of preferences
\item $\uparrow \phi\,\,\,\, \Rightarrow\,\,\,\, \downarrow h_j^{\star}$: since higher values
of $\phi$ indicate a higher preference for leisure, each of the $h_j^{\star}$ is decreasing in $\phi$.
\item conditional on a given value for $\phi$, $h_1 > h_2$
\item cross-sectional variation in hours of work within an occupation is driven by the cross-sectional variation in $\phi$. Because $g''>0$, it is steeper in the
vicinity of $h_1$ than $h_2$. For a given variation in $\phi$, desired hours of work vary more in occupation 2 than in occupation 1.
\end{enumerate}
\end{frame}

\begin{frame}{Qualitative Properties of Equilibrium, Occupation}
\begin{itemize}
\setlength{\itemsep}{0.35 cm}
\item An individual chooses to work in occupation 1 if:
\begin{equation}
\ln(a_1 {h^{\star}}_{1}^{1+\theta})+\phi\frac{(T-{h^{\star}}_1)^{1-\gamma}}{1-\gamma}>\ln(a_2 {h^{\star}}_2)+\phi\frac{(T-{h^{\star}}_2)^{1-\gamma}}{1-\gamma}
\end{equation} 
$\Leftrightarrow$
\begin{align}
& \ln\Big(\frac{a_1}{a_2}\Big)>  \\ \nonumber
&>\underset{\equiv \,\,\,\, z(\phi)}{\underbrace{-(1-\theta)\ln(h^{\star}_1)+\ln(h^{\star}_2)-\phi\Big[\frac{(T-h^{\star}_1)^{1-\gamma}}{1-\gamma}-\frac{(T-h^{\star}_2)^{1-\gamma}}{1-\gamma}\Big]}}
\end{align} 
\item It can be shown that:
\begin{equation}\label{eq:zet}
z'(\phi)=-2\frac{\phi}{1+\theta}(T-h^{\star}_1)^{\gamma}\frac{d h^{\star}_1}{d \phi} > 0
\end{equation}
\end{itemize}
\end{frame}

\begin{frame}{Qualitative Properties of Equilibrium, Occupation}
\begin{enumerate}
\setlength{\itemsep}{0.35 cm}
\item If $\phi \,\,\,\, \indep \,\,\,\, \frac{a_{1}}{a_{2}}$, Eq. (\ref{eq:zet}) implies that individuals with a low taste for leisure ($\phi$ low) are
more likely to work in occupation 1
\item Since hours in occupation 1 are higher than hours in occupation 2 even for a fixed value of $\phi$, this selection effect amplifies the differences in hours worked in the two occupations
\item Since individuals working in occupation 1 are more likely to be selected from those who work long hours, the convexity of g creates a force making the dispersion of hours in occupation 1 small relative to that of occupation 2
\end{enumerate}
\end{frame}

\begin{frame}{Quantitative Assessment}
\begin{itemize}
\setlength{\itemsep}{0.35 cm}
\item Restrict Roy model parameters to be symmetric $(\sigma_{a1}=\sigma_{a1})$ in order to assess the extent to which asymmetry in $\theta$ across occupations can generate quantitative differences in the properties of hours of work across occupations similar to those found in the data
\item Normalize $\overline{\ln a_1}=0$, and pin down $\overline{\ln a_2}$ to target equal employment in the two occupations
\item $T=5200$ hours
\item $\gamma=4$
\item $\theta=0.6$
\end{itemize}
\end{frame}

\begin{frame}{Quantitative Assessment}
Calibration, Targets and Outcomes:
\begin{figure}[!t]
\centering
\begin{minipage}[b]{0.4\textwidth}{}
\centering
\includegraphics[width=1\textwidth]{Calibration_Simple.png}
\end{minipage}
\begin{minipage}[b]{0.55\textwidth}{}
\centering
\includegraphics[width=1\textwidth]{Calibration_Simple_Outcomes.png}
\end{minipage}
\end{figure}
\end{frame}

\begin{frame}{Quantitative Assessment}
The role of selection:
\begin{figure}[!t]
\centering
\begin{minipage}[b]{0.65\textwidth}{}
\centering
\includegraphics[width=1\textwidth]{Calibration_Simple_Selection.png}
\end{minipage}
\end{figure}
Main takeaway: the forces captured in this simple model can provide the basis for a quantitative model of occupational choice and labor supply
\end{frame}

%%%

\section{Multi-Member Households \& Gender Labor Market Outcomes}
\subsection{}

\begin{frame}
\center{\Huge{Multi-Member Households \& Gender Labor Market Outcomes}}
\end{frame}

\begin{frame}{Empirical Facts}
\begin{enumerate}
\item In almost all occupations women work less hours than men and have higher dispersion in hours worked than men
\end{enumerate}
\begin{figure}[!t]
\centering
\begin{minipage}[b]{0.9\textwidth}{}
\centering
\includegraphics[width=1\textwidth]{Hours_Men_Women.png}
\end{minipage}
\end{figure}
\setcounter{ResumeEnumerate}{\value{enumi}}
\end{frame}

\begin{frame}{Empirical Facts}
\begin{enumerate}[start=\numexpr\value{ResumeEnumerate}+1]
\item The distribution of men across occupations in the mean-dispersion space has been relatively stable.
\end{enumerate}
\begin{figure}[!t]
\centering
\begin{minipage}[b]{0.9\textwidth}{}
\centering
\includegraphics[width=1\textwidth]{CDF_men.png}
\end{minipage}
\end{figure}
\setcounter{ResumeEnumerate}{\value{enumi}}
\end{frame}

\begin{frame}{Empirical Facts}
\begin{enumerate}[start=\numexpr\value{ResumeEnumerate}+1]
\item In contrast, there has been a dramatic shift in the distribution of women towards occupations with a higher mean and lower dispersion of hours
\end{enumerate}
\begin{figure}[!t]
\centering
\begin{minipage}[b]{0.9\textwidth}{}
\centering
\includegraphics[width=1\textwidth]{CDF_women.png}
\end{minipage}
\end{figure}
Nevertheless, in 2006-2010 the distribution of women still differs from that of men: women are allocated in occupations with lower mean hours and higher dispersion in hours
\setcounter{ResumeEnumerate}{\value{enumi}}
\end{frame}

\begin{frame}{Empirical Facts}
\begin{enumerate}[start=\numexpr\value{ResumeEnumerate}+1]
\item In most of the occupations, women receive lower hourly wages than men
\end{enumerate}
\begin{figure}[!t]
\centering
\begin{minipage}[b]{0.6\textwidth}{}
\centering
\includegraphics[width=1\textwidth]{Hourly_Wages_men_women.png}
\end{minipage}
\end{figure}
\setcounter{ResumeEnumerate}{\value{enumi}}
\end{frame}

\begin{frame}{Multi-Member Households}
\begin{itemize}
\setlength{\itemsep}{0.35 cm}
\item Extend our Simple Benchmark Model to explicitly include males and females
\item Can trade-offs within the household play an important role in shaping occupational choices and hours worked by gender?
\item Assume an exogenous gender difference in time allocated to non-market activities associated with home production, and assess how this gender asymmetry is propagated to gender asymmetries in occupational choices and the gender wage gap
\end{itemize}
\end{frame}

\begin{frame}{Model}
\begin{itemize}
\setlength{\itemsep}{0.35 cm}
\item There is a unit mass of households, each
composed of a male (\emph{m}) and a female (\emph{f})
\begin{align}
U = U(c_m,l_m,c_f,l_f)&=U(c_m,T_m-h_m,c_f,T_f-h_f)\\ \nonumber
&=\alpha_{m}u_{m}(c_m,T_m-h_m)+\alpha_{f}u_{f}(c_f,T_f-h_f)
\end{align}
where\\
\vfill
$u_g(c_g,T_g-h_g)=\ln c_g + \phi_g \frac{(T_{g}-h_g)^{1-\gamma}}{1-\gamma}, \,\,\,\, \text{for}\,\,\,\, g=m,f$\\
\vfill
$\alpha_{m}+\alpha_{f}=2$
\item $\alpha_g$ are Pareto weights
\item Allow for different endowment of discretionary time, $T_g$
\item Heterogeneity: $(a_{g1},a_{g2},\phi_g)_{g=m,f}\sim$ multivariate log-normal distribution 
\end{itemize}
\end{frame}

\begin{frame}{Competitive Equilibrium}
\begin{itemize}
\setlength{\itemsep}{0.35 cm}
\item Normalize $P_Y=1$ and $A_j=1$ for $j=1,2$
\item Household Opt. Problem:
\begin{equation}
\underset{c_m,h_{jm},\mathbb{I}_m,c_f,h_{jf},\mathbb{I}_f}{\text{max}}\,\,\,\, \Big\{\sum_{g=m}^{f}\alpha_{g}\ln c_{g} + \phi_g \frac{(T_{g}-[\mathbb{I}_{g}h_{g1}+(1-\mathbb{I}_{g})h_{g2})^{1-\gamma}}{1-\gamma} \Big\}
\end{equation}
s.t.
\begin{equation}
\sum_{g=m}^{f}c_g=\Big\{\sum_{g=m}^{f}\mathbb{I}_{g}a_{g1}h_{g1}^{1+\theta_{1}}+(1-\mathbb{I}_g)a_{g2}h_{g2}^{1+\theta_2}\Big\}
\end{equation}
\end{itemize}
\end{frame}

\begin{frame}{Qualitative Properties of Equilibrium, Hours}
\begin{itemize}
\setlength{\itemsep}{0.35 cm}
\item $c^{\star}$: $c_m=\alpha_m\frac{c_m+c_f}{2}$, $c_f=\alpha_f\frac{c_m+c_f}{2}$\\
\item $h_{jg}^{\star}$:
\begin{equation}
\frac{a_{mj}(1+\theta_{j})h_{mj}^{\theta_j}}{a_{mj}h_{mj}^{1+\theta_j}+a_{fl}h_{fl}^{1+\theta_j}}=\alpha_{m}\phi_{m}(T_{m}-h_{mj})^{-\gamma}
\end{equation} 
\begin{equation}
\frac{a_{fl}(1+\theta_{j})h_{fl}^{\theta_j}}{a_{mj}h_{mj}^{1+\theta_j}+a_{fl}h_{fl}^{1+\theta_j}}=\alpha_{f}\phi_{f}(T_{f}-h_{fl})^{-\gamma}
\end{equation} 
where $j$ and $l$ are occupational choices for male and female, respectively
\end{itemize}
\end{frame}

\begin{frame}{Qualitative Properties of Equilibrium, Hours \& Occupation}
\begin{itemize}
\setlength{\itemsep}{0.15 cm}
\item Each HH member equates the disutility of working one more hour to the marginal increase in earnings multiplied by the marginal utility of household consumption
\item Holding the other member’s choices fixed, hours are decreasing in the value of
$\phi$ and choosing the non-linear occupation implies higher hours
\item Cross-effect on hours: the earnings of one spouse have an income effect on the labor supply of the second member
\item Novelty: cross-effect on occupational choice. If one member chooses to work in occupation 1, thereby working longer hours and hence generating more income, this decreases the marginal utility of income earned by the second member and makes it less likely that this individual will work in occupation 1
\end{itemize}
\end{frame}

\begin{frame}{Qualitative Properties of Equilibrium, Hours \& Occupation}
\begin{itemize}
\setlength{\itemsep}{0.35 cm}
\item If women have fewer hours of discretionary time, the direct effect is for them to work fewer hours in the market: a decline in hours is more costly in occupation 1 than in occupation 2
\item These selection effects serve to increase the average quality of females in occupation 1 and decrease the average quality of females in occupation 2
\item Through intra-household effects, these effects on female choices impact male choices, which in turn amplify the direct effects on female choices.
\end{itemize}
\end{frame}

\begin{frame}{Quantitative Analysis}
\begin{itemize}
\setlength{\itemsep}{0.35 cm}
\item Restrict Roy model parameters to be symmetric $(\sigma_{a1}=\sigma_{a1})$ in order to assess the extent to which asymmetry in $\theta$ across occupations can generate quantitative differences in the properties of hours of work across occupations similar to those found in the data
\item $T_m=5200$, $T_f=4700$ hours; 10 hours/week difference
\item $a_{gj}$, $\phi_{g}$ do not vary across males \& females except for a uniform (across occupations) skill gap
\item $\rho_{a1,\phi}=\rho_{a2,\phi}=0$, $\rho_{am1,af1}=\rho_{am2,af2}=0$ 
\end{itemize}
\end{frame}

\begin{frame}{Quantitative Analysis, Calibration}
\begin{figure}[!t]
\centering
\begin{minipage}[b]{0.65\textwidth}{}
\centering
\includegraphics[width=1\textwidth]{Multimember_Calibration.png}
\end{minipage}
\end{figure}
\end{frame}

\begin{frame}{Quantitative Analysis, Calibration}
\begin{itemize}
\setlength{\itemsep}{0.25 cm}
\item calibration yields a positive correlation
between skills ($\rho_{a1,a2}$ = 0.344) and a higher variance in ability 1 than ability 2
($\sigma^2_1 = 0.315 > \sigma^2_
2 = 0.195$). This leads more individuals with higher absolute advantage to
choose occupation 1
\item Gender skill gap (or wage discrimination): $\mu_{ma}-\mu_{fa}=0.33$
\item Females have lower Pareto Weights $(\alpha_f=0.773)$
\item $\rho_{ma,fa}=0.655>\rho_{mw,fw}$
\item In order to get $\rho_{h_{m},h_{f}}=0.02$ it must be that $\rho_{\phi_{m},\phi_{f}}=0.755$
\end{itemize}
\end{frame}

\begin{frame}{Quantitative Analysis, Model Fit}
\begin{figure}[!t]
\centering
\begin{minipage}[b]{0.65\textwidth}{}
\centering
\includegraphics[width=1\textwidth]{Multimember_Fit.png}
\end{minipage}
\end{figure}
\end{frame}

\begin{frame}{Quantitative Analysis, Model Fit}
\begin{itemize}
\setlength{\itemsep}{0.25 cm}
\item The calibration achieved the Gender Wage Gap with an \emph{exogenous} skill gap component of 33\%, implying that the \emph{endogenous} component of the gender wage gap is 9\%. Thus, the mechanisms present in our model that we sought to assess quantitatively account for about 25\% of the overall gender wage gap in the data.
\end{itemize}
\end{frame}

\begin{frame}{Quantitative Analysis, Model Features}
\begin{figure}[!t]
\centering
\begin{minipage}[b]{0.65\textwidth}{}
\centering
\includegraphics[width=1\textwidth]{Multimember_Features.png}
\end{minipage}
\end{figure}
\end{frame}

\begin{frame}{Quantitative Analysis, Model Features}
\begin{itemize}
\setlength{\itemsep}{0.25 cm}
\item Everyone Single, with Gender Skill Gap $\Rightarrow$ Gender Wage Gap
\item Random Matching (no Time Endowments difference)
\begin{itemize}
\item Gender Gap in dispersion of Hours Worked
\item Gender Gap in Occupational Choice
\item The lower hours of work of women relative to men creates a disincentive for choosing the non-linear occupation, causing women to move from the non-linear to the linear sector
\end{itemize}
\end{itemize}
\end{frame}

\begin{frame}{Quantitative Analysis, Model Features}
\begin{itemize}
\setlength{\itemsep}{0.25 cm}
\item Assortative Matching
\begin{itemize}
\item more similarity among household members makes their choices more substitutable: the gender wage gap, the gender hours gap and the gender occupational choice gap all increase
\item wealth effect: small increase in hours even for women and a small increase in the fraction of women who work in the non-linear occupation
\end{itemize}
\item Heterogeneous Pareto Weights: decreasing the relative weight on the utility of females implies less weight on their leisure, thereby leading to a reallocation of working time from men to women
\end{itemize}
\end{frame}

\begin{frame}{Quantitative Analysis, Model Features}
\begin{itemize}
\setlength{\itemsep}{0.25 cm}
\item Heterogeneous Time Endowments
\begin{itemize}
\item Female hours drop
\item Make hours increase
\item Gender occupational gap increases: the large decrease in desired hours of work for women leads to a large decrease in hours in the non-linear occupation
\item The women remaining in the non-linear (moving to the linear) occupation are then more positively (negatively) selected in terms of their skills
\item Gender wage gap increases
\end{itemize}
\end{itemize}
\end{frame}

\begin{frame}{Quantitative Analysis, Selection}
\begin{figure}[!t]
\centering
\begin{minipage}[b]{0.6\textwidth}{}
\centering
\includegraphics[width=1\textwidth]{Multimember_Selection.png}
\end{minipage}
\end{figure}
\end{frame}

\begin{frame}{Quantitative Analysis, Conclusion}
\begin{itemize}
\setlength{\itemsep}{0.25 cm}
\item Family interactions play a very important role in propagating underlying sources of gender asymmetries
\item The propagation via family interactions is influenced in a very significant way by the extent of assortative matching
\item The endogenous component of the gender wage gap in our baseline model
reflects the propagation of three underlying sources of gender asymmetries: the uniform skill gap, the discretionary time gap and the Pareto weight gap. The
effects of the 10\% gender gap in discretionary time is the largest
\end{itemize}
\end{frame}

%\begin{frame}<0>
%\addtocounter{framenumber}{-1} 
%\scriptsize
%\bibliographystyle{apalike}
%\bibliography{References}
%\end{frame}


%\begin{appendix}
%
%%\section{Appendix}
%%\subsection{}
%%
%%\begin{frame}
%%\center{\Huge{Appendix}}
%%\end{frame}
%
%
%\end{appendix}

%\begin{frame}<0>
%\addtocounter{framenumber}{-1} 
%\scriptsize
%\bibliographystyle{ecta}
%\bibliography{References}
%\end{frame}
%\bibliographystyle{ecta}
%\bibliography{References}

\end{document}
